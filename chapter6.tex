\chapter{Styling}
\label{sec:ch6}

Less is used in place of CSS, to provide syntactic sugars\footnote{"Syntactic sugar" is a term for syntax changes in computer programming which make it easier for humans to code.} for us to make our lives easier, including nested styles and variables. 

Note that all knowledge you have learnt for CSS can be transferable to Less, because Less is a superset of CSS. However, we will start from scratch for those of you who have not learnt CSS before.

\section{Further Resources (Ch 6)}

I didn't have this piece of notes back when I first learned Less. Here is \href{https://youtu.be/YD91G8DdUsw}\footnote{Link: \url{https://youtu.be/YD91G8DdUsw}{the video}} that I used to learn the basics. 

You could refer to the \href{https://www.w3schools.com/cssref/}{w3schools documentation}\footnote{Link: \url{https://www.w3schools.com/cssref/}} to learn how to use some more styling tools.

\section{Referencing elements using classes and IDs}

In this section we will investigate ways to reference the elements we created in the Pug files, so that we can add styles for them.

The easiest way is to reference them using tags, the style below applies to all \texttt{h1} tags in the whole web page. Each style is followed by a semi-colon, and curly braces are used to surround all the styles for the same tag.

\begin{lstlisting}[language=pug]
h1{
    color: red;
}
\end{lstlisting}

However, making all the \texttt{h1} tags red is not usually what we want. To style for specific elements, we would add a class or an ID to the element in the pug file, then reference the class or ID in our less file.

For example, referring to the code from the \hyperref[sec:classesids]{previous chapter on classes and IDs}, we can write styles to center align the self introduction, and highlight specified items in the shopping list yellow and italic.

\begin{lstlisting}[language=pug]
#self-intro{
    text-align: center;
}
.warning{
    color: yellow;
    font-style: italic;
}
\end{lstlisting}

We would explain the effect and syntax of different styles in detail in later sections. The difference between using IDs and classes would be explained in the \hyperref[sec:nestedstyles]{nested styles session}.

\section{Text}

\subsection{Text colour}

This changes the colour of the text. Use the \texttt{color}\footnote{Sorry it has to be American spelling :(} keyword, followed by a colon, then specify your colour before finishing off with a semi-colon.

The colour can be specified in \href{https://www.rapidtables.com/web/color/RGB_Color.html}{RGB colour codes}\footnote{Choose your colour: \url{https://www.rapidtables.com/web/color/RGB_Color.html}}, e.g. \texttt{\#FFFFFF} indicates white. \href{https://www.w3.org/wiki/CSS/Properties/color/keywords}{Common colours}\footnote{Full list of "common" colours: \url{https://www.w3.org/wiki/CSS/Properties/color/keywords}} can be specified in text for your convenience, as shown in the previous section.

\begin{lstlisting}
h1{
    color: #FFFF00;
}
\end{lstlisting}

\subsection{Background colour}

\texttt{background-color} sets the colour of the background. The default value is \texttt{transparent}. It is important to distinguish between a white background and a transparent background. More on background colour in \hyperref[sec:margin]{the section on margin and paddings}.

\begin{lstlisting}[language=pug]
#self-intro{
    background-color: #777; //#777 is equivalent to #777777
}
\end{lstlisting}

\subsection{Text alignment}

\texttt{text-align} sets the alignment of text. It accepts values \texttt{left} (default), \texttt{center}\footnote{Note the American spelling}, \texttt{right}, and \texttt{justify}\footnote{Sometimes \texttt{justify} might not work very well, if so just use \texttt{left} instead.}

\begin{lstlisting}[language=pug]
#self-intro{
    text-align: right;
}
\end{lstlisting}

\section{Positioning}

\subsection{Width}

\subsection{Height}

\subsection{Flexbox}

\section{Margins, paddings and borders}
\label{sec:margin}

\subsection{Margins}

\subsection{Paddings}

\subsection{Borders}

\section{Other styles}

\subsection{Bold}

\subsection{Italics}

\section{Nested styles}
\label{sec:nestedstyles}

\subsection{Styling priority}

\section{Using variables}