\chapter{Bootstrap}
% Old Ch. 7

\section{Logistics}

Bootstrap makes your website prettier in the following ways.

\begin{itemize}
\item Providing default styles (\cref{sec:bootstrapbasic})
\item Providing class short hands to smoothen your workflow (\cref{sec:classshorthands})
\item Providing tools to achieve \textbf{mobile-first} styling, such as the \textbf{grid system} (\cref{sec:styling}, \cref{sec:grid})
\item Providing 
out-of-the-box components for you to use (\cref{sec:components})
\end{itemize}

I recommend you checking out the \href{https://getbootstrap.com/docs/5.2/getting-started/introduction/}{Bootstrap documentation}\footnote{Link: \url{https://getbootstrap.com/docs/5.2/getting-started/introduction/}} when you are making websites of your own, see if any of their components suit your needs. Using them would definitely save you a lot of work! We will also rely on the documentation throughout this chapter because they are very good.

I am using Bootstrap version 5.2 but it would certainly update, please follow the latest documentation, I hope there are no breaking changes if you are reading this in the distant future.\footnote{I was using Bootstrap V4 around 4 years ago back in 2018}

\subsection*{Installation}
\label{sec:bootstrapbasic}

\subsection*{Basic styling provided by Bootstrap}

\subsection*{Containers}
\label{sec:container}



\section{Class short hands}
\label{sec:classshorthands}

\section{Components}
\label{sec:components}

In the following two sections, we would go over how to use two of the Bootstrap components in detail. The idea is after we have done so, you would be able to understand the general way of using all Boostrap components by reading the documentation on your own.

What I would do first is to check out all the components in the list and find out which component suits the best for the look I envisioned. (Or if there is none that might help then we can write our own styles like in \cref{sec:styling})

\subsection{Carousel}
\label{sec:carousels}

The carousel is a slideshow for cycling through a series of images, that seems quite a fit for the index page. So I go to the \href{https://getbootstrap.com/docs/5.2/components/carousel/}{documentation page of the carousel}\footnote{Link: \url{https://getbootstrap.com/docs/5.2/components/carousel/}}.



% TODO IMAGES of components documentation showing the list of components as well
\subsection{Card}
\label{sec:cards}
