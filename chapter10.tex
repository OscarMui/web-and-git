\chapter{Misc}
\label{sec:misc}

\textit{Of less importance}
\vspace{6mm}

\section{Fonts}

\section{Web Hosting}

\section{Optional QOL tools for MacOS}
\label{sec:iterm}
\textit{Of less importance, covered in \href{https://www.youtube.com/watch?v=ZIBEVGrtiVA&list=PLjGmdnqrOKuYXiu7lgG5HW71jPEUd1XCm&index=7}{video 6a of the series}}
\vspace{6mm}

Refer to \href{https://www.youtube.com/watch?v=ZIBEVGrtiVA&list=PLjGmdnqrOKuYXiu7lgG5HW71jPEUd1XCm&index=7}{the video}\footnote{Link: \url{https://www.youtube.com/watch?v=ZIBEVGrtiVA&list=PLjGmdnqrOKuYXiu7lgG5HW71jPEUd1XCm&index=7}}.

Here is an outline on what is installed.
\begin{itemize}
    \item iTerm2 - \url{https://iterm2.com/}
    \item oh my zsh - \url{https://ohmyz.sh/}
    \item Colour theme - \url{https://github.com/MartinSeeler/iterm2-material-design}, follow the documentation in the README and apply the colour theme to your terminal
    \item Changing the theme of oh my zsh by editing the \texttt{.zshrc} file by running \texttt{open .zshrc}. Edit one of the lines to \texttt{ZSH\textunderscore THEME = "ys"} 
    \item tldr pages - \url{https://tldr.sh/}

\end{itemize}

\section{Optional VS Code Plugins}
\textit{Of less importance}
\vspace{6mm}

Refer to the very last section of \href{https://github.com/OscarMui/setup-cheetsheet-macos}{my setup cheetsheet}\footnote{Link: \url{https://github.com/OscarMui/setup-cheetsheet-macos}} (last part works for all operating systems) Not all of them are useful to you, just choose ones that you like.

Here is an outline on something that might be useful. These are the extension IDs, copy and paste them in the search extensions box to find them.

\begin{itemize}
    \item sdras.night-owl (A cooler theme)
    \item aaron-bond.better-comments (Coloring comments with //!, //TODO, //?)
    \item bierner.markdown-preview-github-styles (Preview of .md files) 
    \item adpyke.codesnap (Beautiful screenshots, suggest checking codesnap.transparentBackground)
    \item mhutchie.git-graph (View Git history by Git Graph: view Git Graph (git log))
\end{itemize}