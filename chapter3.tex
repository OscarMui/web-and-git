\chapter{Git}

Git is important because it allows you to work and communicate with people using just a few simple commands. By following its rules and conventions, you can work with others efficiently, without having to send zip files back and forth through emails. 

\begin{center}
\includegraphics[width=10cm]{images/ch0-version-control.jpeg}
\end{center}

\section{Git VS GitHub}

Before we start, it is a good idea to distinguish between Git and GitHub. Git runs on your local machine, keeping track of the changes you have made locally. While GitHub is a cloud service provider\footnote{There are also other similar cloud service provider but GitHub is the most popular one}, keeping track of changes made to different people. That version of the project on GitHub should be the one that all local machines follow.

\begin{center}
\includegraphics[width=10cm]{images/ch3-gitgithub.png}
\end{center}

\section{First time using Git}
\label{sec:gitfirst}

\textit{Covered in \href{https://www.youtube.com/watch?v=wQmFz-Ggxuo&list=PLjGmdnqrOKuYXiu7lgG5HW71jPEUd1XCm&index=4}{video 3 of the series}}
\vspace{6mm}

Let's just change out \texttt{app/templates/views/index.pug} slightly so that we can make our first push to GitHub. Or if you are working on other projects, you can pick a random file and made some edits.

\begin{lstlisting}[language=pug]
//- app/templates/views/index.pug
extends ../layouts/default

block content
	.container
		h1 Welcome to Static Web!
		br
		p A very cool website.
		p This web is made for tutorial purposes.
\end{lstlisting}

One very useful command - \texttt{git status} displays information about current situation, it might also contains some hints on which command(s) you need to run to proceed.

\begin{lstlisting}[language=bash]
$ git status
On branch main
Your branch is up to date with 'origin/main'.

Changes not staged for commit:
  (use "git add <file>..." to update what will be committed)
  (use "git restore <file>..." to discard changes in working directory)
	modified:   app/templates/views/index.pug

no changes added to commit (use "git add" and/or "git commit -a")
\end{lstlisting}

\subsection*{First add}
So you can see it is prompting you to run \texttt{git add <file>}. Instead, I would prefer \texttt{git add .} As discussed in \hyperref[sec:dir]{the previous chapter}, \texttt{.} refers to the current directory. That means, we are adding all the changes in the folder to be ready for a commit.

\begin{lstlisting}[language=bash]
$ git add .
\end{lstlisting}

Then, run \texttt{git status} again to asset the situation. The modified file should now be highlighted in green. (sadly you cannot see the green highlighting in LaTeX)

\begin{lstlisting}[language=bash]
$ git status
On branch main
Your branch is up to date with 'origin/main'.

Changes to be committed:
  (use "git restore --staged <file>..." to unstage)
	modified:   app/templates/views/index.pug
\end{lstlisting}

\subsection*{First commit}

Next step, we are now ready to commit. \texttt{git commit} seals and confirms your changes into one chunk, and adds a commit message. The \textbf{commit message} should be meaningful, summarising what you have achieved with these code changes.

\begin{lstlisting}[language=]
$ git commit -m "experimenting with git"
[main e7bc529] experimenting with git
Committer: KidProf <kidprof@KidProfs-MacBook-Air.local>
Your name and email addresses were configured automatically
...
git config --global --edit
git config --amend --reset-author
1 file changed, 2 insertions(+), 1 deletion(-)
\end{lstlisting}

We got a problem here, our username and email do not match with the ones on GitHub. We need to \#1) change the default name and email on our device, and \#2) change the name and email for this particular commit.

Instead of the commands recommended automatically by Git, I recommend variants of those commands which are listed below, because these commands won't trigger the command line text editor to pop up. \textit{\hyperref[sec:vim]{(more in chapter 2)}}

Run the following two commands for \#1, replace with your own ones. Surround your name with double quotes if your name contains spaces.

\begin{lstlisting}[language=bash]
$ git config --global user.name KidProf
$ git config --global user.email kidprof@gmail.com
\end{lstlisting}

To verify, rerun those two commands but without the name and email, it should display your name and email as output.

\begin{lstlisting}[language=bash]
$ git config --global user.name
KidProf
$ git config --global user.email
kidprof@gmail.com
\end{lstlisting}

This command is for \#2.

\begin{lstlisting}[language=bash]
$ git config --amend --reset-author --no-edit
\end{lstlisting}

All set, now try running \texttt{git status} again.

\begin{lstlisting}[language=]
$ git status
On branch main
Your branch is ahead of 'origin/main' by 1 commit.
  (use "git push" to publish your local commits)

nothing to commit, working tree clean
\end{lstlisting}

\subsection*{First push}

Time to run \texttt{git push}, our last step. Except it returns an error.

\begin{lstlisting}[language=]
$ git push
fatal: The current branch main has no upstream branch.
To push the current branch and set the remote as upstream, use

    git push --set-upstream origin main
\end{lstlisting}

This one is quite an easy fix, just run the substitute command as instructed. You will need this whenever you are pushing a new branch to GitHub.

\begin{lstlisting}[language=]
$ git push --set-upstream origin main
Enumerating objects: 11, done.
Counting objects: 100% (11/11), done.
Delta compression using up to 8 threads
Compressing objects: 100% (6/6), done.
Writing objects: 100% (6/6), 575 bytes | 575.00 KiB/s, done.
Total 6 (delta 3), reused 0 (delta 0), pack-reused 0
remote: Resolving deltas: 100% (3/3), completed with 3 local objects.
To github.com:KidProfMC/tutorial-website.git
\end{lstlisting}

\begin{center}
\includegraphics[width=12cm]{images/ch3-firstpushsuccess.png}
\end{center}

There you go, that is the first push. I will not list out the results of \texttt{git status} anymore in future sections, because they are taking too much space, but bear in mind running \texttt{git status} regularly is very useful.

\section{A typical day using Git}
\label{sec:gcmsg}

\textit{Covered in \href{https://www.youtube.com/watch?v=wQmFz-Ggxuo&list=PLjGmdnqrOKuYXiu7lgG5HW71jPEUd1XCm&index=4}{video 3 of the series}}
\vspace{6mm}

It is much more regular after the first push. You just need to remember three steps, add, commit and push.

\begin{lstlisting}[language=bash]
$ git add .
$ git commit -m "your message here"
$ git push
\end{lstlisting}

\texttt{git add} adds the changes you have made to be ready for the commit. 

\texttt{git commit} seals and confirms your changes into one chunk, and adds a commit message. The \textbf{commit message} should be meaningful, summarising what you have achieved with these code changes.

\texttt{git push} uploads the commit from your local machine to GitHub. \textbf{Push} is another fancier word for upload
\vspace{6mm}

You should perform these three steps whenever you have made some progress on your code, as if it is a save button. Practice makes perfect!

\section{Visualising}
\label{sec:sublime}
Git is a command line tool and is best used through the command line. Again, you will come across some situations where only command line is available (e.g. when accessing a remote server). 

However, it is acceptable if you just want to visualise the git commit history. Just don't run any commands using them. The following are two software recommended by me to do so. Sublime Merge looks nicer while Git Graph is visible inside VS Code. 

\subsection*{Sublime Merge}

Download Sublime Merge by following \href{https://www.sublimemerge.com/}{this link}.\footnote{Link: \url{https://www.sublimemerge.com/}} It should be straightforward. 

After installing, open the repository folder. The commit history is displayed.

\begin{center}
\includegraphics[width=15cm]{images/ch3-sublimemerge.png}
\end{center}

\subsection*{Git Graph}

Git Graph is a VS Code extension, download it inside VS Code.

\begin{center}
\includegraphics[width=13cm]{images/ch3-gitgraph0.png}
\end{center}

Press \texttt{cmd+shift+P} or \texttt{ctrl+shift+P} to open the VS Code command bar, type \texttt{Git Graph: View Git Graph}

\begin{center}
\includegraphics[width=13cm]{images/ch3-gitgraph1.png}
\end{center}

The commit history is displayed.

\begin{center}
\includegraphics[width=13cm]{images/ch3-gitgraph2.png}
\end{center}

\section{Something went wrong?} 

Anytime if you are unsure about what is happening with Git, \textbf{\texttt{git status}} is your best friend! It will display information about current situation, it might also contains some hints on which command(s) you need to run to get out of nasty situations.
\vspace{6mm}

If you added a files that you don't want to push, and you have not committed, you can try \texttt{git restore <file>} to revert the changes for a particular file, or \texttt{git restore .} to revert all changes.

However, if you made some irreversible/ hard to reverse mistakes, which everybody will at some point in time. For example, you have made a commit at the wrong branch but haven't pushed, or you failed to fix a merge conflict. 

In these cases, let's give up and use a cheesy way to fix it. Move all your code to another location so that you remember what you have changed, clone the repository from GitHub again, then apply your changes to the new copy of the code, but with the mistakes fixed. It is completely fine, and sometimes the easiest way to do something.

\section{Repositories}

Again, repository is a fancier name for project. Now we will look closely at this concept, and explain how to establish the relationship between your local code and the GitHub code.

\subsection{Creating a repository on GitHub}
\label{sec:gituploadgithub}

\textit{\textbf{WARNING: } You can only do so when the repository is empty. If the GitHub repository is not empty, you will need to create a new one.}

It is as simple as pressing a "New" button on the GitHub website, then fill in some simple details, such as your repository name and whether you would like your repository to be publicly visible or private, the decision is yours.

You can add a \hyperref[sec:readme]{\texttt{README.md}} or \hyperref[sec:gitignore]{\texttt{.gitignore}}, the usage of such files will be explained in the \hyperref[sec:projstructure]{next chapter}. However, it would make the repository no longer empty and you will have trouble \hyperref[sec:gituploadgithub]{uploading local code to the new repository.}

\subsection{Uploading existing local code to a new GitHub repository}

Use \texttt{git init} in the folder with all your projects.

The use of \texttt{git branch -M main} is to change the name of the default branch from "master" to "main". A decision that GitHub made in 2020.\footnote{More info: \url{https://www.jumpingrivers.com/blog/git-moving-master-to-main/}}

Then follow your normal routine of adding and committing \hyperref[sec:gcmsg]{as above}. There is a convention naming the first commit "init" or "initial commit".

Then \texttt{git remote add origin} is to establish the linkage between the local code and the GitHub repository, so that Git knows where to push to when you run \texttt{git push}, except for the first time, you need to run \texttt{git push -u origin main}.

\begin{lstlisting}[language=bash]
# KidProf in ~/code/your-project
$ git init

# No need to remember, copy from the GitHub docs
$ git branch -M main 

$ git add .
$ git commit -m "init"

# No need to remember, copy from the GitHub docs
$ git remote add origin git@github.com:KidProfMC/git-practice.git
$ git push -u origin main
\end{lstlisting}

\begin{figure}[h]
\centering
\includegraphics[width=15cm]{images/ch3-newrepocode.png}
\caption{The documentation on GitHub serves you well}
\end{figure}

\subsection{Downloading code from GitHub to local device}

Use \texttt{git clone}, then a new folder with the repository name would be created with all the code inside. This is the method that we used in \hyperref[sec:install7]{chapter 1}.

\begin{lstlisting}[language=bash]
# KidProf in ~/code
$ git clone git@github.com:KidProfMC/git-practice.git
$ cd git-practice
\end{lstlisting}

\subsection{\texttt{.git} folder}

The \texttt{.git} folder contains everything that git needs to know to function normally, the commit history, the GitHub repository that it is linked to, etc.. If you remove the \texttt{.git} folder from your local device through the file explorer or finder (you need to enable the setting "show hidden files and folders"), or by the command \texttt{rm -rf .git}, all git information would be deleted.

