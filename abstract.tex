\chapter*{Preface}

This piece of notes covers one way of building a static website. It facilitates my live teaching sessions or my Web \& Git YouTube series.
\vspace{6mm}

We use Pug.js (Sometimes abbreviated to Pug in this piece of notes) instead of HTML, less instead of CSS. And used gulp to translate the files back to HTML and CSS. The generated HTML files can be directly opened by the browser. We also included Bootstrap to make our static website responsive.

At the same time, we will be learning how to use Git and command line tools.

\section{About me}

I am a second-year undergraduate student studying Computer Science at the University of Oxford.

\section{Prerequisites}

Basic coding experience (e.g. variables, loops) in any programming languages (e.g. C++, Python) is expected. Knowing basic HTML and CSS would be helpful, but not a must. I would contrast HTML with Pug.js and CSS with less throughout the notes, if you have not learned HTML and CSS you could skip those sessions.

\section{Way to read the notes}

I think this piece of notes is like a dictionary. You should not read it in sequence, but read the related sections when you want to learn more about the topic or if my teaching isn't detailed enough. 
\vspace{6mm}

However, I understand that some of you might want to understand the material without either attending my course or watching my teaching videos, so I marked the less essential sections with \textit{Advanced} or \textit{Of less importance}. You can read the sections without the markings first, then those with \textit{Advanced}, then those with \textit{Of less importance}.

\section{Rationale}

I would like to explain what I would like to achieve this piece of notes in this section, so as to manage your expectations.
\vspace{6mm}

\textbf{TL;DR: I believe what you will learn is useful in some way if you would like to pursue jobs related to IT, or dig deeper in programming and Computer Science. Knowledge in Git is also crucial for Maths and Science undergraduates as they also need to write code sometimes as a group.}

\subsection*{Why making websites?}

A website is cross-platform, meaning that it can be opened on different kinds of devices, basically any device with a browser. From mobile phones, no matter it uses Android or iOS or whatever; to laptops and desktops, no matter it uses Windows or MacOS or Linux or whatever.

In contrast, you can only code mobile apps for one specific platform\footnote{Of course you can use Flutter or React Native, but they have their downsides}, Android apps cannot be installed on iOS devices. 

\subsection*{Command line and Git}

Learning how to make a website is just a facade. The most important thing that you will take away is the use of command line and Git. Command line is important because you can work more efficiently if you are proficient at command line tools, you will come across some situations where only command line is available (e.g. when accessing a remote server). 

\subsection*{Why not plain HTML?}

Pug.js allows the use of templates, hence there is no need to modify every page file whenever we need to something that every page has in common. Pug.js also allows the use of variables, so that we can control what to be rendered based on the situation, hence providing a basis for building dynamic websites.\footnote{Dynamic websites are those that could interact with backend servers and databases securely. A framework that you can learn after mastering this piece of notes is ExpressJS. Link: \href{https://expressjs.com/}{https://expressjs.com/}}

\subsection*{Responsiveness}

Mobile devices are common nowadays, so our website has to cater for screens with different sizes, from those as small as mobile phones to those as big as projectors. 

A website is responsive if it can rearrange its elements for easy readability based on the screen size and orientation. The use of Bootstrap makes our website responsive.

\section{A word of warning}

This is just a draft, aiming to include everything in the shortest amount of time possible, so explanations and examples may be inadequate. If there are any errors in the notes feel free to contact me by email oscar.mui@univ.ox.ac.uk

\section{Linktree}

Video series:

\href{https://www.youtube.com/playlist?list=PLjGmdnqrOKuYXiu7lgG5HW71jPEUd1XCm}{https://www.youtube.com/playlist?list=PLjGmdnqrOKuYXiu7lgG5HW71jPEUd1XCm}
\vspace{0mm} % SPECIAL

Example website:

\href{https://numbersarefun.netlify.app/}{https://numbersarefun.netlify.app/}
\vspace{6mm}

Template for you to start from scratch:

\href{https://github.com/KidProf/static-web-sandbox}{https://github.com/KidProf/static-web-sandbox}
\vspace{6mm}

Source code of the finished website:

\href{https://github.com/KidProf/numbersarefun-sample-temp}{https://github.com/KidProf/numbersarefun-sample-temp}
\vspace{6mm}

Pug.js: 

\href{https://pugjs.org/}{https://pugjs.org/}
\vspace{6mm}

Less: 

\href{https://lesscss.org/}{https://lesscss.org/}
\vspace{6mm}

Bootstrap: 

\href{https://getbootstrap.com/}{https://getbootstrap.com/}
\vspace{6mm}

